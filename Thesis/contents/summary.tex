% !TEX root = ../main.tex

\chapter{总结与展望}
\label{summary}

\section{本文总结}

随着集群技术的发展,在现代空战中,大规模集群空战是各个主要军事大国争夺的科技制高点。本文以空空导弹集群为研究目标,考虑了导弹的预计攻击时间和机动消耗能量,将未来的态势因素纳入模型考虑之中,建立了空空导弹任务分配模型。本文以博弈论为理论工具,使用势博弈模型、偏好联盟博弈模型研究了静态分布式任务分配方法,并以随机博弈模型为基础探讨了动态分布式任务分配的解决方案,取得了一些成果,本文的工作和主要贡献如下:

1. 针对分布式静态任务分配问题,本文基于势博弈模型理论设计了任务效用函数和基于WLU的智能体效用函数,针对现有使用势博弈模型解决任务分配问题的文献大多没有考虑含约束分配问题,本文引入Lagrange乘子对WLU效用函数进行修正,并从理论上证明了修正后的WLU效用函数的可行性,推导出了使用修正WLU效用函数的势博弈模型的性能下界。接着,结合现有文献成果,本文介绍了三种博弈学习算法:JSFP、GRMFMI和SAP算法,并针对博弈学习算法容易过早收敛,陷入局部最优的问题,在SAP算法的基础上设计了任务交易机制,提出了含任务交易的SAP算法。通过与几种集中式启发算法的对比仿真验证,结果表明本文设计的模型与算法在平衡分配问题和非平衡分配问题上都具有良好的效果,且相比集中式算法在较大规模问题上更具时间优势。

2. 将静态任务分配问题看作联盟形成问题,本文基于偏好联盟博弈理论建立了任务分配模型。在偏好联盟博弈模型中,引入智能体对联盟的偏好关系,并假设智能体对联盟的偏好仅与联盟成员数量有关,而与具体成员无关,并从理论上证明当智能体偏好关系满足社交疏远性质时,偏好联盟博弈模型一定存在纳什均衡。本文根据以上结论设计了边际效用递减的任务效用函数和满足社交疏远性质的智能体效用函数,并针对具体函数推导出偏好联盟博弈模型的性能下界。针对博弈学习算法,本文除了使用势博弈模型中的带任务交易的SAP算法外,还介绍了一种分布式一致算法DMCA,用于实现分布式架构下的联盟形成问题。通过设计仿真试验,本文对比分析了非平衡分配问题中两种非平衡程度对于学习算法的性能影响,结果表明本文提出的两种博弈模型下的五种学习算法在非平衡问题中具有良好的求解能力,并且有着良好的实时性。

3. 针对动态任务分配模型,本文基于随机博弈理论,将动态任务分配模型分解为若干时间窗口下的静态博弈阶段,在每个阶段博弈中建立势博弈或偏好联盟博弈模型并求解。针对博弈阶段之间的切换问题,设计了初始分配生成机制和分配传递机制,以保证在满足动态要求的前提下,保持分配的稳定性和可行性。针对动态环境中可能出现的非理想情况,本文针对非连通场景和目标数量变化两种场景设计了重分配触发机制,分析了非连通情况下可以触发重分配和消解分配冲突的条件,并针对几种目标数量变化的场景下的重分配进行了分析。本文建立了完整的分布式动态任务分配系统,并通过仿真实验验证了该系统的有效性,并设计了几种非理想情况下的重分配场景,验证了系统在突发事件下具有良好的动态自适应性。

\section{未来展望}

本文基于博弈论的思想对空空导弹的分布式任务分配模型和算法进行了相关研究,但实际场景中大规模多对多空战场景要更为复杂,分配难度也更大,且本文提出的模型和算法也存在着一些局限与问题还需要进一步的研究。下一步工作主要包括以下几个方面:

1. 动态任务分配环境的建模问题。本文建立的任务分配问题模型是基于导弹预计剩余攻击时间和预计机动能量,虽然一定程度上考虑了未来的因素,但实际模型更加复杂,包括导弹集群的路径规划、目标机动、导弹和目标的运动特性等因素对分配的影响尚未考虑在内。如何选取对任务分配有影响的因素,并处理这些因素的动态变化对分配的影响,结合导弹与目标的运动规律,建立更符合实际的任务分配模型是一个仍待解决的问题。

2. 算法的精确度和普适性问题。基于博弈论的任务分配模型和算法可以在较快时间内获得较为精确的解,但受限于模型的设置,这类算法往往比集中式算法更容易陷入局部最优中,因此也导致了求解结果较为分散的问题,。针对以上问题,本文虽然提出了一些解决思路,但并没有完全解决。此外,基于博弈论的分配模型受限于模型函数和参数的设计,设计不当可能会使得模型在某些场景下失效,本文针对模型的设计和参数的选择范围进行了初步探究,但这个问题仍需更多的工作。

3. 动态任务重分配机制设计问题。本文只研究了通信网络非连通和目标数量变化两种场景下的重分配问题,但实际动态场景中,需要进行重分配的事件更多更复杂,比如导弹发生故障、目标突然丢失等等;此外,当重分配事件增多时,如何协调这些事件的优先级和处理方式本文没有深入展开研究。

